%!TEX root = ../thesis.tex

\section{Discussion}
\label{sec:discussion}

\subsection{Scope of the Model}
\label{sec:model-scope}

In summary, the MINRV8 architecture knows three groups of instructions (cf. table \ref{tbl:min-arch-instrs}):
\begin{itemize}
    \item Computational instructions such as Mov, \minrv{And}, \minrv{Add}, etc.
    \item Memory instructions \minrv{Load} and \minrv{Store}
    \item System instructions \minrv{Ecall}, \minrv{Mret}, \minrv{Csrrs}, \minrv{Csrrc}
\end{itemize}

MINRV8 is meant to be a reasonable abstraction of a real-world \gls{riscv} architecture from a security standpoint.
However, up to this point nothing has been said about the limits of this abstraction.
With every abstraction, there is a small chance that it perfectly matches the concept that has been abstracted but in most cases, some corners are cut.
In this section, we will reflect on the limits of the MINRV8 architecture and its implementation.

% TODO: mention that other fields are well modelled
% TODO: mention that we actually don't need all computational instructions
% TODO: mention performance hit of dynamic memory regions

A reader, experienced in the field of microcontrollers or computer-architecture in general might wonder why our model does not include:
\begin{enumerate}
    \item Executable memory and a \gls{pc}
    \item Jump or branch instructions
    \item A model of a stack pointer
\end{enumerate}

In the introduction of the MINRV8 architecture in section \ref{sec:minrv8} we mentioned that the idea of the architecture was tightly coupled to its implementation.
The design of the model had to answer the question: How can a stream of instructions be implemented?
nuXmv allows to think of the following options:
\begin{enumerate}
    \item \label{itm:exmem-frozen}
    Use \smv{FROZENVAR}s to model executable memory - a \smv{FROZENVAR} is something like a constant in other programming languages but without a fixed value.
    nuXmv chooses the value on the first simulation step but does not change it afterwards.
    This is more efficient than using plain \smv{VAR}s with no transitions.
    \item \label{itm:exmem-var}
    Use \smv{VAR}s to model executable memory.
    In practice, this would mean that the memory of the implementation as described in section \ref{sec:model-implementation} would be much larger than the current 4 bytes.
    \item \label{itm:exmem-ivar}
    Finally, use \smv{IVAR}s to model the stream of instruction.
    This is the option we decided to go for as described in section \ref{sec:model-implementation}.
    Using \smv{IVAR}s means that there is no model of executable memory.
    % TODO: is the part about the ALU actually true?
    Instead, the input variables provided to the implementation model the result of an instruction decoder that hands the \gls{alu} the ready-to-use decoded instructions on each transition of the simulated model.
    As such, the architecture does not need to worry \textit{where} these instructions come from.
\end{enumerate}

\subsection{Trustworthiness of the Model}

\subsection{Trustworthiness of the Verification Process}

\subsection{Limitations of the Results}
