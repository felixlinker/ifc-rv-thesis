%!TEX root = ../thesis.tex

\section{Introduction}
\label{sec:introduction}

This thesis falls in the realm of formal verification, i.e. the attempt to verify a system by the use of \textit{formal methods} such as SAT solvers, interactive theorem provers and model checkers.
Here, the approaches of two papers are combined:

\citeauthor{Reid17} in \citetitle{Reid17} \cite{Reid17} stressed the need of verifying specifications themselves as opposed to simply verifying implementations against specifications.
To tackle this, he proposed to verify specifications against higher level properties and used this methodology to verify the specification of the \gls{arm} M-class specification.
In this thesis, two lines of his research will be continued: \begin{enumerate*}[label=\alph*)]
    \item the main contribution of this thesis also is to propose an approach for verifying specifications against higher level properties and
    \item this methodology will be applied to an \gls{isa} specification as well.
\end{enumerate*}

The idea behind the higher level properties an \gls{isa} will be verified against mainly stems from the work of \citeauthor{Ferraiuolo17} in \citetitle{Ferraiuolo17} \cite{Ferraiuolo17}.
As the title indicates, their work also falls into the domain of verification; the authors propose a new way of verifying \gls{hdl} code implementations of \glspl{isa}.
The core idea behind this is to enhance the type system of the existing \gls{hdl} language \gls{secverilog}.
The types available to this type system serve as security annotation of information.
By applying typing rules to expressions in the \gls{hdl} code, new labels are generated and thus information is tracked through the code.
Certain type conversions are, however, prohibited and mark a security vulnerability.
Thus, their approach can be summarized as tracking and controlling information flow in \gls{hdl} code.
They evaluate their approach by applying it to the implementation of the \gls{trustzone} extension.
In this thesis, the idea of tracking and controlling information flow as proposed in \cite{Ferraiuolo17} will be lifted to the level of the \gls{isa} specification thus following the line of research as proposed in \cite{Reid17}.

This approach will be implemented using a model checker.
This allows for enhancing on the approach proposed by \citeauthor{Reid17} \cite{Reid17} in a key aspect:
In his work, \citeauthor{Reid17} focused on higher lever properties to verify the \gls{arm} \gls{isa} against that were limited to making specifications about a single transition of the processor only, i.e. that only take the pre- and post-state of a single cycle of the processor into account.
By using a model checker, it is possible to consider sequences of instructions, i.e. sequences of processor-transitions, of unbounded length.

% TODO: Add motivation and goal of the thesis
% TODO: Introduce structure of thesis
