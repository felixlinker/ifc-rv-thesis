\documentclass{securem}
\usepackage[a4paper]{geometry}
\usepackage[utf8]{inputenc}
\usepackage{enumitem}
\usepackage{csquotes}
\usepackage{tikz}
\usepackage[hidelinks]{hyperref}
\usepackage{syntax}
\usepackage[euler]{textgreek}
\usetikzlibrary{positioning,calc,graphs}

\setlength{\grammarindent}{8em}

% fonts
\usepackage{tgpagella}
\usepackage{courier}

\newcommand{\muZ}{{\textmu}Z}

\title{Master-thesis Secure-M \\ \large{An outline}}
\author{Felix Linker}
\date{}

\sloppy

\begin{document}

\maketitle

\section{Introduction}

In \cite{Ferraiuolo17} the authors present an approach to formally verifying the ARM TrustZone architecture by using static information flow analysis.
They apply the method of information flow control (IFC) to the hardware description language (HDL) implementing said architecture.
Their approach mainly consists of two concepts: A lattice of security labels and type-system rules that define how information flows through the HDL code.
Intuitively, it is allowed for information to flow "up" the lattice but not "down".

The goal of this project is to apply this approach of using security labelings to model information flow to the process of model checking an architectural model of a microprocessor with multiple privilege levels.
The model checker will quantify over programs running on the microprocessor and check properties like: "It is not possible for secure information stored at point Y to leak".
Such "points" at which information might be stored include registers and memory of the micro-controller.
In a 32-bit architecture, this gives us $ (R \cdot 32 + M \cdot 8)^2 $ implicit properties to check where $ R $ is the number of registers and $ M $ is the number of bytes in memory - for each bit in registers and memory, we can check whether information can leak to any other bit in registers or memory.
We expect to find programs that will violate such properties simply because privileged code can do \textit{anything} that is also leaking information.

This approach must not be seen as an attempt to formally verify the security of an architecture itself.
It is an attempt to find program patterns that can leak information and finding rules that - when complied with - guarantee the absence of such leaks.
We will however also have in mind that a IFC formalism might not be able to find \textit{all} issues that can arise during the programming of micro-controllers.
Therefore another goal of this project is to explore the power of a labeled-IFC approach.

\section{Background}

\subsection{RISC-V}

RISC-V (RISC stands for \textbf{R}educed \textbf{I}nstruction \textbf{S}et \textbf{C}omputer) is a modular processor architecture specification.
RISC-V is specified by two documents: the RISC-V User Level ISA \cite{RiscVISA} and the RISC-V Privileged Architectures manuals \cite{RiscVISAP}.
The User Level ISA describes the basic functionality of the processor which is divided into several modules, e.g. integer arithmetic, floating point arithmetic, debugging, that can freely be combined with each other.
The only necessity for each RISC-V CPU is to implement the integer arithmetic module.

The modularity of the RISC-V architecture allows implementers to easily build very simple and small processors, that still are fully compliant with the specification.

In this thesis, we will model check the RISC-V architecture.
We will use the base integer instruction set, the standard extension for user-level interrupts and the privileged architecture to model a micro-controller that has all basic instructions needed to perform computational tasks, support for interrupts and exceptions and two or more privilege modes - thus all function necessary to have sufficient complexity in computing but not too many features to make our undertaking infeasible.

\subsection{nuXmv}

nuXmv \cite{Cavada14} is a symbolic model checker a predecessor of which was introduced in \textit{NuSMV: A New Symbolic Model Checker} \cite{Cimatti2000}.
nuXmv extends NuSMV with new data types being integers and reals and enhances on its SMT-based model checking techniques.

nuXmv can check linear temporal logic (LTL) formulas, computational tree logic (CTL) formulas, invariants and property specification language (PSL) formulas for unbounded models.
An example of such a model is given in snippet \ref{snpt:nuxmv}.
The input language to nuXmv knows modules as the highest tier language objects.
An entry point to a nuXmv file is given by the module called \texttt{main}.
In the example, the module has two states the active of which is kept in a dedicated variable and a result variable called \texttt{res}.
There also is an input variable \texttt{d} that is assigned with a random real number on each step of the system\footnote{%
    Random in this case means non-deterministically in regards to the modules state.
    nuXmv will choose values with some determinism when trying to disprove specifications of the system.
}
The \texttt{ASSIGN}, \texttt{INIT} and \texttt{TRANS} expression describe the system by logical formulas.
nuXmv will try to prove that each specified formula (in this case an invariant) holds for all automata induced by the description given on all execution paths.

\begin{smv}[caption={An example for the nuXmv input language \cite{Cavada14}},label={snpt:nuxmv}]
MODULE main
IVAR
    d : real;
VAR
    state : {s0, s1};
    res : real;
ASSIGN
    init(state) := s0;
    next(state) := case
        state = s0 & res >= 0.10 : s1;
        state = s1 & res >= 0.20 : s0;
        TRUE                     : state;
    esac;
    next(t) := case
        state = s0 & res < 0.10 : res + d;
        state = s1 & res < 0.20 : res + d;
        TRUE                    : 0.0;
    esac;
INIT
    res >= 0.0;
TRANS
    (state = s0 -> (d >= 0 & d < 0.01)) &
    (state = s1 -> (d >= 0 & d <= 0.02));
INVARSPEC res <= 0.3;
\end{smv}

We will use nuXmv to describe the model we want to check the information flow for.
We will build a custom model of the RISC-V architecture in the nuXmv input language and phrase properties about information flow for this model in either LTL, CTL or as invariants.

\section{Project Plan}

The project divides itself into the following steps:
\begin{enumerate}
    \item Development of a simplified architectural model which should include
    \begin{itemize}
        \item Protected system control registers
        \item General purpose registers
        \item A model of memory with secure and non-secure regions
        \item At least 8 instructions: load immediate, load, store, add, subtract, compare, branch, enter privileged mode, leave privileged mode
    \end{itemize}
    \item Formal implementation of the architectural model in the nuXmv input language
    \item Verification
    \item Iteration \& refinement
    \begin{itemize}
        \item What kind of bad programming patterns can be distinguished?
        \item What kind of rules can prohibit such bad patterns?
        \item Can positive effects of these rules be proven?
        \item Does the formal model need to be refined?
    \end{itemize}
    \item Evaluation
\end{enumerate}

\bibliography{references}
\bibliographystyle{alpha}

\end{document}
