\documentclass{securem}
\usepackage[a4paper]{geometry}
\usepackage[utf8]{inputenc}
\usepackage{enumitem}
\usepackage{csquotes}
\usepackage{tikz}
\usepackage[hidelinks]{hyperref}
\usepackage{syntax}
\usepackage[euler]{textgreek}
\usetikzlibrary{positioning,calc,graphs}

\setlength{\grammarindent}{8em}

% fonts
\usepackage{tgpagella}
\usepackage{courier}

\newcommand{\muZ}{{\textmu}Z}

\title{Master-thesis Secure-M \\ \large{An outline}}
\author{Felix Linker}
\date{}

\sloppy

\begin{document}

\maketitle

\section{Introduction}

In \cite{Ferraiuolo17} the authors present an approach to formally verifying the ARM TrustZone architecture by using static information flow analysis.
They apply the method of information flow control (IFC) to the hardware description language (HDL) implementing said architecture.
They approach mainly consists of two things: A lattice of security labels and type-system rules that define how information flows through the HDL code.
Intuitively, it is allowed for information to flow "up" the lattice but not "down".

The goal of this project is to apply this approach of using security labellings to model information flow to the process of model checking an architectural model of a microprocessor with multiple privilege levels.
The model checker will quantify over programs running on the microprocessor and check properties like: "It is not possible for secure information stored at point Y to leak".
Such "point" at which information might be stored, include registers and memory of the micro-controller.
In a 32-bit architecture, this gives us $ (R \cdot 32 + M \cdot 8)^2 $ implicit properties to check where $ R $ is the number of registers and $ M $ is the number of bytes in memory - for each bit in registers and memory, we can check whether information can leak to any other bit in registers or memory.
We expect to find programs that will violate such properties simply because privileged code can do \textit{anything} that is also leaking information.

This approach must not be seen as an attempt to formally verify the security of an architecture itself.
It is an attempt to find program patterns that can leak information and finding rules that - when complied with - guarantee the absence of such leaks.
We will however also have in mind that a IFC formalism might not be able to find \textit{all} issues that can arise during the programming of micro-controllers.
Therefore another goal of this project is to explore the power of a labeled-IFC approach.

\section{Background}

\subsection{RISC-V}

RISC-V (RISC stands for \textbf{R}educed \textbf{I}nstruction \textbf{S}et \textbf{C}omputer) is a modular processor architecture specification.
There are many modules, e.g. integer arithmetic, floating point arithmetic, debugging, that can freely be combined with each other.
The only necessity for each RISC-V CPU is to implement the integer arithmetic module.

We will use parts of the base integer instruction set and privileged architecture in this project to model a micro-controller that has all function necessary to have sufficient complexity in computing but not too many features to make our undertaking infeasible.

\subsection{$ \mu $Z \& Datalog}

$ \mu $Z is a Datalog engine that comes with the SMT-solver Z3 from Microsoft.
It can solve Horn-clauses and queries whether certain rules are reachable.
The syntax of Datalog is very much comparable to the syntax of the programming language Prolog.

We will use $ \mu $Z to perform the model checking process.

\subsection{SPACER}

SPACER is an algorithm that uses a Datalog engine to perform automated abstraction on Horn-clauses expressed in propositional Linear Rational Arithmetic and was proposed in \cite{Komuravelli13}.
We might alter this algorithm in this project to be applicable to the theory of bit vectors and use it in our model checking procedure.

\section{Project Plan}

The project divides itself into the following steps:
\begin{enumerate}
    \item Development of a simplified architectural model which should include
    \begin{itemize}
        \item Protected system control registers
        \item General purpose registers
        \item A model of memory with secure and non-secure regions
        \item At least 8 instructions: load immediate, load, store, add, subtract, compare, branch, enter privileged mode, leave privileged mode
    \end{itemize}
    \item Formal implementation of the architectural model in SMT
    \item Verification
    \begin{itemize}
        \item Verification can be done straight-forward but it might be advisable to use algorithms like SPACER to enhance performance and output
    \end{itemize}
    \item Iteration \& refinement
    \item Evaluation
\end{enumerate}

\bibliography{references}
\bibliographystyle{alpha}

\end{document}
