\begin{tabular}{| l | r c c c c | c c c c c |}
    \hline
    & & \minrv{0} & \minrv{1} & \minrv{2} & \minrv{3} & \minrv{p} & \minrv{0c} & \minrv{0p} & \minrv{1c} & \minrv{1p} \\
    \hline
    
     & \minrv{r} & \PU & \PU & \PU & \PU & \multirow{2}{*}{\minrv{U}} & \multirow{2}{*}{\minrv{UN}}  & \multirow{2}{*}{\minrv{-R-}} & \multirow{2}{*}{\minrv{WB}} & \multirow{2}{*}{\minrv{L--}} \\ & \minrv{m} & \PT & \PT & \PT & \PT &&&&& \\
    \hline
    
    \multirow{2}{*}{\minrv{Store 0, 2}} & \minrv{r} &  &  &  &  &  &   &  &  &  \\ & \minrv{m} &  &  &  &  &&&&& \\
    \hline
    
    \multirow{2}{*}{\minrv{Slt 2, 0, 0}} & \minrv{r} &  &  &  &  & \multirow{2}{*}{\minrv{M}} &   &  &  &  \\ & \minrv{m} &  &  &  &  &&&&& \\
    \hline
    
    \multirow{2}{*}{\minrv{San}} & \minrv{r} & \PT & \PT & \PT & \PT &  &   &  &  &  \\ & \minrv{m} &  &  &  &  &&&&& \\
    \hline
    
    \multirow{2}{*}{\minrv{Load 2, 0}} & \minrv{r} &  &  & \PU &  &  &   &  &  &  \\ & \minrv{m} &  &  &  &  &&&&& \\
    \hline
    
    \multirow{2}{*}{\minrv{Csrrc 1, 0, 2}} & \minrv{r} &  & \CT &  &  &  &   &  & \multirow{2}{*}{\minrv{UN}} &  \\ & \minrv{m} &  &  &  & \PU &&&&& \\
    \hline
    
    \multirow{2}{*}{\minrv{Csrrc 3, 3, 2}} & \minrv{r} &  &  &  &  &  &   &  &  &  \\ & \minrv{m} &  &  &  &  &&&&& \\
    \hline
    
\end{tabular}

